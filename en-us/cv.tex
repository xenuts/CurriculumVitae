\documentclass[10pt]{article}

\usepackage{marvosym}
\usepackage{fontspec}
\usepackage{xunicode,xltxtra,url,parskip}
\defaultfontfeatures{Scale=MatchLowercase, Mapping=tex-text}
\RequirePackage{color,graphics}
\usepackage[usenames,dvipsnames]{xcolor}
\usepackage[left=1in, right=1in, top=0.5in, bottom=0.5in]{geometry}
\usepackage{supertabular}
\usepackage{titlesec}
\usepackage{multicol}
\usepackage{multirow}
\usepackage{longtable}
\usepackage{xstring}
\usepackage{ifthen}
\usepackage{enumitem}
\usepackage[none]{hyphenat}
\usepackage[xetex,
            pdfauthor={Xiangzeng Zhou},
%            pdftitle={Xiangzeng Zhou Résumé},
            pdftitle={Xiangzeng Zhou CV},
            pdfsubject={Xiangzeng Zhou CV},
            pdfkeywords={object detection, object tracking, pattern recognition, machine learning},
            pdfproducer={xelatex},
            pdfcreator={xelatex}]{hyperref}
\usepackage[absolute]{textpos}
\usepackage{enumitem}
\usepackage{tabularx, array}

\newcommand{\xphoto}{\includegraphics[scale=0.75]{../../../Images/xzzhou-id3.jpg}}
%%%%%%%%%% 导入中文环境 %%%%%%%%%%
%\usepackage{fontspec} %使可以設定字型
%\usepackage{xeCJK} %讓中英文字型分開設置
%\setmainfont{Courier New} % 设置英文衬线字体
%\setCJKmainfont{宋体} % 設定中文字型,英文仍為 TeX 字型
%\XeTeXlinebreaklocale "zh" % 這行及下一行使中文能自動換行
%\XeTeXlinebreakskip = 0pt plus 1pt
%
%%%%%%% 设置中文字号 %%%%%%
%\newcommand{\chuhao}{\fontsize{42pt}{\baselineskip}\selectfont}     % 初号
%\newcommand{\xiaochuhao}{\fontsize{36pt}{\baselineskip}\selectfont} % 小初号
%\newcommand{\yihao}{\fontsize{28pt}{\baselineskip}\selectfont}      % 一号
%\newcommand{\erhao}{\fontsize{21pt}{\baselineskip}\selectfont}      % 二号
%\newcommand{\xiaoerhao}{\fontsize{18pt}{\baselineskip}\selectfont}  % 小二号
%\newcommand{\sanhao}{\fontsize{15.75pt}{\baselineskip}\selectfont}  % 三号
%\newcommand{\sihao}{\fontsize{14pt}{\baselineskip}\selectfont}      % 四号
%\newcommand{\xiaosihao}{\fontsize{12pt}{\baselineskip}\selectfont}  % 小四号
%\newcommand{\wuhao}{\fontsize{10.5pt}{\baselineskip}\selectfont}    % 五号
%\newcommand{\xiaowuhao}{\fontsize{9pt}{\baselineskip}\selectfont}   % 小五号
%\newcommand{\liuhao}{\fontsize{7.875pt}{\baselineskip}\selectfont}  % 六号
%\newcommand{\qihao}{\fontsize{5.25pt}{\baselineskip}\selectfont}    % 七号

% Template obtained from http://www.cv-templates.info/2009/03/professional-cv-latex/

%Setup hyperref package, and colours for links
\definecolor{linkcolour}{rgb}{0,0.2,0.6}
\hypersetup{colorlinks,breaklinks,urlcolor=linkcolour, linkcolor=linkcolour}

%Color
\definecolor{lightg}{HTML}{999999}
\definecolor{medg}{HTML}{666666}
\definecolor{darkg}{HTML}{333333}

% Bullets
\definecolor{noteone}{HTML}{999999}
\definecolor{notetwo}{HTML}{848484}
\definecolor{notethree}{HTML}{424242}
\definecolor{notefour}{HTML}{212121}
\definecolor{notefive}{HTML}{000000}

\newcommand{\fivenotes}{%
	\textcolor{noteone}{\symbol{"2022}}
	\textcolor{notetwo}{\symbol{"2022}}
	\textcolor{notethree}{\symbol{"2022}}
	\textcolor{notefour}{\symbol{"2022}}
	\textcolor{notefive}{\symbol{"2022}}
}
\newcommand{\fournotes}{%
	\textcolor{noteone}{\symbol{"2022}}
	\textcolor{notetwo}{\symbol{"2022}}
	\textcolor{notethree}{\symbol{"2022}}
	\textcolor{notefour}{\symbol{"2022}}
	\textcolor{white}{\symbol{"2022}}
}
\newcommand{\threenotes}{%
	\textcolor{noteone}{\symbol{"2022}}
	\textcolor{notetwo}{\symbol{"2022}}
	\textcolor{notethree}{\symbol{"2022}}
	\textcolor{white}{\symbol{"2022}}
	\textcolor{white}{\symbol{"2022}}
}
\newcommand{\twonotes}{%
	\textcolor{noteone}{\symbol{"2022}}
	\textcolor{notetwo}{\symbol{"2022}}
	\textcolor{white}{\symbol{"2022}}
	\textcolor{white}{\symbol{"2022}}
	\textcolor{white}{\symbol{"2022}}
}
\newcommand{\onenote}{%
	\textcolor{noteone}{\symbol{"2022}}
	\textcolor{white}{\symbol{"2022}}
	\textcolor{white}{\symbol{"2022}}
	\textcolor{white}{\symbol{"2022}}
	\textcolor{white}{\symbol{"2022}}
}

\newcommand{\oneskill}{%
  \textcolor{white}{\symbol{"2022}}
  \textcolor{white}{\symbol{"2022}}
  \textcolor{notefive}{\symbol{"2022}}
}

\newcommand{\twoskill}{%
  \textcolor{white}{\symbol{"2022}}
  \textcolor{notethree}{\symbol{"2022}}
  \textcolor{notefive}{\symbol{"2022}}
}

\newcommand{\threeskill}{%
  \textcolor{noteone}{\symbol{"2022}}
  \textcolor{notethree}{\symbol{"2022}}
  \textcolor{notefive}{\symbol{"2022}}
}

%FONTS
 \defaultfontfeatures{Mapping=tex-text}
 \setmainfont{Adobe Garamond Pro}

%\setromanfont [Ligatures={Common}, BoldFont={Linux Libertine Bold}, ItalicFont={Linux Libertine Italic}]{Linux Libertine}
%\setsansfont [Ligatures={Common}, BoldFont={GeosansLight}, ItalicFont={GeosansLight}]{GeosansLight}
%\setmonofont{GeosansLight}

\font\lighttext=''Helvetica:color=787878'' at 10pt
\font\lighttextweb=''Helvetica:color=FF1493'' at 10pt

%CV Sections inspired by:
%http://stefano.italians.nl/archives/26
\titleformat{\section}{\Large\scshape\raggedright\sffamily}{}{0em}{}[\titlerule]
%% \titlespacing{\section}{0pt}{-2pt}{0pt}

%-------------WATERMARK TEST [**not part of a CV**]---------------
\TPGrid[30mm,30mm]{30}{60}
%\setlength{\TPHorizModule}{30mm}
%\setlength{\TPVertModule}{\TPHorizModule}
%\textblockorigin{2mm}{0.65\paperheight}
\setlength{\parindent}{0pt}

\newcommand{\skill}{\textbf}
\newcommand{\institution}{\textsc}

% \def\bullet{\textcolor{medg}{\symbol{"00BB}}}
\def\div{\,\textbar{}\,}

\usepackage{xspace}

% language macros

\newcommand{\lang}[2]{\expandafter\def\csname #1\endcsname{\skill{#2}\xspace}}

\lang{js}{JavaScript}
\lang{python}{Python}
\lang{java}{Java}
\lang{matlab}{MATLAB}
\lang{bash}{Bash}
\lang{c}{C}
\lang{cpp}{C++}
\lang{ccpp}{C/C++}
\lang{scheme}{Scheme}
\lang{django}{Django}
\lang{numpy}{NumPy}
\lang{scipy}{SciPy}
\lang{opencv}{OpenCV}
\lang{jquery}{jQuery}
\lang{git}{git}
\lang{linux}{Linux}
\lang{android}{Android}
\lang{html}{HTML/CSS}
\lang{visb}{Visual Basic}
\lang{haskell}{Haskell}
\lang{php}{PHP}
\lang{sql}{SQL}

\titlespacing{\section}{0pt}{-2pt}{0pt}

\begin{document}
\pagestyle{empty}
\par{ \centering{\Huge Curriculum Vitae} \bigskip\par}

\vspace{2em}

\begin{multicols}{2}
  \setlength{\parskip}{0pt}
  \section{Personal Details}
  \begin{tabularx}{\linewidth}{@{}l X@{}}
    \textsc{Surname}	    & \normalsize{Zhou} \\
    \textsc{First Name}	    & \normalsize{Xiangzeng} \\
    \textsc{Date of Birth}  & \normalsize{April 18 1988}\\
    \textsc{Occupation}     & \normalsize{Ph.D. Student} \\
    \textsc{Mobile}         & \normalsize{+086-15934898828}
  \end{tabularx}

  \begin{picture}(0, 0)(-397, -62) %\begin{picture}(0, 0)(-159, -5)
    \put(0, 0){\xphoto}  %\put(0,0){\color{red}\line(1,0){30}}
  \end{picture}

  \vfill \columnbreak
  \section{Contacts}
  \begin{tabularx}{\linewidth}{@{}l X@{}}
    \textsc{Affiliation}	& \normalsize{School of Computer Science, Northwestern Polytechnical University (NPU)} \\
    \textsc{Address}      & \normalsize{No.127 West Youyi Road, Xi’an, China, 710072} \\
    \textsc{Email}        & \normalsize{\href{mailto:xzzhou@nwpu-aslp.org}{xzzhou@nwpu-aslp.org},\ \ \href{mailto:xenuts@gmail.com}{xenuts@gmail.com}}
  \end{tabularx}
\end{multicols}

\section{Research Interests}
\normalsize{Object Tracking, Object Recognition, Machine Learning, Deep Learning}

%%%%%%%%%%%%%%%% Qualifications %%%%%%%%%%%%%%%%%%%%%%%%%%%%%
\newcommand{\EduEntry}[4]{\textsc{#1} & \textbf{#2} & \textsc{#3} & \textbf{#4}\\}
\newcommand{\ExpEntry}[4]{
  \multirow{2}{1.1cm}[1pt]{\textsc{#1}} & \multicolumn{2}{l}{\textbf{#2}} \\
  \nopagebreak & #3 & \multirow{1}{9.5cm}[75pt]{#4} \\
  \nopagebreak \multicolumn{3}{c}{} \\ [-2ex]
}
 
\vspace{0.8em}
\section{Qualifications}
\begin{tabular*}{\textwidth}{@{\extracolsep{\fill}}r l p{5.5cm} r}
  \EduEntry{Jul. 2010 - Present}%
  {Ph.D.}%
  {Computer Science \& Technology}%
  {Northwestern Polytechnical University}

  \EduEntry{Sept. 2006 - Jul. 2010}%
  {B.S. }%
  {Computer Science \& Technology}%
  {Northwestern Polytechnical University}
\end{tabular*}

%%%%%%%%%%%%%%%% Project Experience %%%%%%%%%%%%%%%%%%%%%%%%%%%%%
% \newcommand{\ExpEntryS}[6]{
% \textsc{#1}  & \textbf{#2} #3 \textsc{#4}\\
% \nopagebreak & \multicolumn{2}{p{5.5in}}{\small{#5}}\\
% \nopagebreak & \multicolumn{2}{l}{#6} \\
% \nopagebreak \multicolumn{3}{c}{} \\ [-1ex]
% }

% \newcommand{\ExpEntryL}[5]{
% \textsc{#1} & \textbf{#2} #3 \textsc{#4}\\
% \nopagebreak &\multicolumn{2}{p{5.5in}}{\small{#5}} \\
% }

\vspace{0.8em}
\section{Project Experience}
\setlength\LTleft{0pt}
\setlength\LTright{0pt}
\vspace{-0.5em}
\begin{longtable}{@{\extracolsep{\fill}} l | l l }
  \ExpEntry{2013.06 $\sim$ 2014.03}
  {Object Tracking using Deep Learning Technology}
  {\includegraphics[height=30mm]{../../../Images/edt-450x275.png}}
  {\raggedright We tackle the generic object tracking problem by a novel approach that incorporates a deep learning architecture with an on-line AdaBoost framework. Inspired by its multi-level feature learning ability, a stacked denoising autoencoder (SDAE) is used to learn multi-level feature descriptors from a set of auxiliary images.}

  \ExpEntry{2012.04 $\sim$ 2012.12}
  {Ball Trajectory Tracking in Tennis Game Video}
  {\includegraphics[height=30mm]{../../../Images/tennis-450x275.png}}
  {A two layered data association method to improve the robustness of tennis ball tracking. At the local layer, a shift token transfer method is proposed, based on shift window processing, to generate a set of short trajectories or ``trajectorylets''. At the global layer, a unique ball trajectory is obtained by applying a dynamic programming based splice method to a directed acyclic graph consisting of trajectorylets.}  
  \ExpEntry{2011.05 $\sim$ 2011.12}
  {Towards a Queue-Aware ATM: Monitoring and Managing Queues in front of ATMs}
  {\includegraphics[height=30mm]{../../../Images/atm-450x275.png}}
  {In order to monitoring queues in front of ATMs, apply stereo camera real-time object tracking approach. With the aid of camera’s real-time tracking, develop a simple application system which can give customer a suggested queue and estimated queuing time.}
  
  \ExpEntry{2010.09}
  {Keyword Spotting based Real-time Dialog System}
  {\includegraphics[height=30mm]{../../../Images/best-450x275.png}}
  {Make a question-set limited Dialog System implemented by Keyword Spotting approach with improved Online Garbage Model. My main task is proposing a improved method based on Online Garbage Keyword Spotting Model.}

  \ExpEntry{2010.09}
  {Keyword Spotting Tool}
  {\includegraphics[height=30mm]{../../../Images/keyword-450x275.png}}
  {Make a Keyword Spotting Tool which user can dynamically add or remove keywords by hand.} 
\end{longtable}

% \begin{longtable}{@{\extracolsep{\fill}} l | l r}
%   \ExpEntryS{2013.6 - 2014.3}%  
%   {Object Tracking using Deep Learning Technology}%
%   {} {} {aasf as sfasdf asf asdfas asdf asf w fsef t ewrtf asgsdfasdf asdf sa sdf patterns for english, usenglishmax, dumylang, nohyphenation, farsi, arabic, croatian, bulgarian, ukrainian, russian.}
% {}%  {\includegraphics[width=45mm]{edt.png}}
  
%   \ExpEntryS{2012.4 - 2012.12}%  
%   {Ball Trajectory Tracking in Tennis Game Video}%
%   {} {} {}
%   {\includegraphics[width=45mm]{tennis.png}}
  
%   \ExpEntryS{2011.5 - 2011.12}%  
%   {Towards a Queue-Aware ATM: Monitoring and Managing Queues in front of ATMs}%
%   {} {}
%   {In order to monitoring queues in front of ATMs, apply stereo camera real-time object tracking approach. With the aid of camera’s real-time tracking, develop a simple application system which can give customer a suggested queue and estimated queuing time.}
%   {\includegraphics[width=45mm]{atm.jpg}}

%   \ExpEntryS{2010.9}%
%   {Keyword Spotting based Real-time Dialog System}%
%   {} {}
%   {Make a question-set limited Dialog System implemented by Keyword Spotting approach with improved Online Garbage Model. My main task is proposing a improved method based on Online Garbage Keyword Spotting Model.}
%   {\includegraphics[width=45mm]{bestknown.jpg}}

%   \ExpEntryS{2010.9}%
%   {Keyword Spotting Tool}%
%   {} {} 
%   {Make a Keyword Spotting Tool which user can dynamically add or remove keywords by hand.}
%   {\includegraphics[width=45mm]{keyword.jpg}}
% \end{longtable}

%%%%%%%%%%%%%%%%%% Skills \& Languages %%%%%%%%%%%%%%%%%%%%%%%%%%%

\newcommand{\SkillEntry}[2]{ \item #2 #1 }
\vspace{-0.5em}
\section{Skills}
\vspace{-1em}
\begin{multicols}{4}
\raggedcolumns
\begin{itemize}
\renewcommand{\labelitemi}{}
\renewcommand{\skill}{\textnormal}
\setlength{\itemsep}{1pt}
\setlength{\parskip}{0pt}
\setlength{\parsep}{0pt}

\SkillEntry{C}{\threeskill}
\SkillEntry{Python}{\twoskill}
\SkillEntry{Matlab}{\threeskill}
\SkillEntry{Bash}{\twoskill}
\SkillEntry{English}{\twoskill}
\SkillEntry{Emacs}{\twoskill}
\SkillEntry{OpenCV}{\threeskill}
\SkillEntry{Git}{\twoskill}
\SkillEntry{Linux}{\twoskill}
\SkillEntry{\LaTeX}{\twoskill}
\end{itemize}
\end{multicols}\vspace{-1em}

%  \begin{footnotesize}
%    \oneskill Small-scale projects and/or assignments \hfill
%    \twoskill Multiple projects and/or experience teaching \hfill
%    \threeskill Large-scale and/or multi-group projects
%  \end{footnotesize}


%%%%%%%%%%%%%%%%%%%%%%%%%%%%%%%%%%%%%%%%%%%%%

%\newcommand{\proj}[3]{
%  \textsc{#1} & #2\\
%   &\href{http://www.#3}{#3}\\
%   \multicolumn{2}{c}{} \\ [-1ex]
%}
%\newcommand{\projl}[3]{
%  \textsc{#1} & #2\\
%   &\href{http://www.#3}{#3}\\
%}
%\newcommand{\projlh}[4]{
%  \textsc{#1} & #2\\
%   &\href{#3}{#4}\\
%}
%\section{Personal and Open Source Projects}
%\begin{tabularx}{\textwidth}{@{}p{3cm}|X@{}}
%\proj{\href{http://nonpartisan.me}{nonpartisan.me}}%
%{Google Chrome extension that filters social media websites for political keywords.  Available in the \href{https://chrome.google.com/webstore/detail/nonpartisanme/ninebcppidndhampaggnjbijpacoadgg}{Chrome Web Store}.  Featured in the \href{http://www.charlestoncitypaper.com/charleston/sick-of-politics-on-facebook-try-this-browser-tool/Content?oid=4153447}{Charleston City Paper}.}%
%{github.com/malloc47/nonpartisan.me}
%
%\proj{term-do}{An interactive terminal prompt that displays potential command completions as you type.  A hybrid of gnome-do and Emacs's ido-mode.  Works on many tested VT100 terminal types; built in~\cpp.  Includes client/server architecture implemented with boost.interprocess and full-featured plugin system. Available in the \href{https://aur.archlinux.org/packages/term-do-git/}{Arch Linux AUR}.}{github.com/malloc47/term-do}
%\end{tabularx}


%%%%%%%%%%%%%%%%% Honors/Awards  %%%%%%%%%%%%%%%%%%%%%

\vspace{0.8em}
\section{Honors and Awards}
\begin{tabularx}{\textwidth}{@{}r|X l|p{4.9cm}@{}}
2014 & The Paper of ICIP 2014 was recognized as the Top 10 papers \\
2014 & Received IEEE Signal Processing Society Travel Grant to Attend ICIP 2014, Paris, France \\
2013 & Received IEEE Signal Processing Society Travel Grant to Attend ICASSP 2013, Vancouver, Canada \\
2011 & Awarded by Northwestern Polytechnical University Scholarship Fund for Six-month Visiting Researcher \\
2010 & First Prize Scholarship of Northwestern Polytechnical University \\
2009 & National Endeavor Scholarship \\
2009 & First Prize of C Programming Contest of Northwestern Polytechnical University \\
2009 & First Prize Scholarship of Northwestern Polytechnical University \\
2008 & National Endeavor Scholarship \\
2008 & First Prize Scholarship of Northwestern Polytechnical University \\
2007 & Second Prize of ACM Programming Contest of Northwestern Polytechnical University \\
2007 & First Prize Scholarship of Northwestern Polytechnical University \\
2007 & Third Prize of Mathematical Modeling Contest of Northwestern Polytechnical University
\end{tabularx}

%%%%%%%%%%%%%%%%% Activities %%%%%%%%%%%%%%%%%%%%%%%%%%%%

\vspace{0.8em}
\section{Activities}
\begin{tabularx}{\textwidth}{@{}r|X l|p{4.9cm}@{}}
Oct. 28, 2014            &  \textbf{Oral Presentation} on ICIP 2014 \\
May 30, 2013             &  \textbf{Poster Presentation} on ICASSP 2013 \\
Mar., 2012 - Sept., 2012 &  \textbf{Visiting Researcher} at University of East Anglia, Norwich, U.K. \\
Apr., 2011 - Oct., 2011  &  \textbf{Conference Organizing Committee Member} for APSIPA ASC 2011 \\
Nov. 23 - 25, 2010       &  \textbf{Oral Presentation} on ICALIP 2010 \\
Oct. 26 - 29, 2010       &  \textbf{Invited Demonstration} for UIC/ATC 2010 \\
Jul., 2009               &  \textbf{Intern} at China Pacific Insurance (Group) Co., Ltd. \\
Jun., 2009               &  \textbf{Intern} at KunShan (Suzhou) Ambow Software Training Base
\end{tabularx}

%%%%%%%%%%%%%%%%% Publications %%%%%%%%%%%%%%%%%%%%%%%%%%%%

\let\originalbibitem\bibitem
\def\bibitem#1#2\par{%
  \noexpandarg
  \originalbibitem{#1}
  \StrSubstitute{#2}{Xiangzeng Zhou}{\textbf{Xiangzeng Zhou}}\par}

\nocite{Zhou-2014-Ensemble}
\nocite{Zhou-2014-Tennis}
\nocite{Zhou-2013-Two}
\nocite{Huang-2012-Detection}
\nocite{Li-2011-Realtime}
\nocite{Niu-2011-Multiconfidence}
\nocite{Xie-2010-Speech}

\renewcommand\refname{Publications}
{\footnotesize \bibliography{Xenuts-Bibliography}}
\bibliographystyle{plainyr-rev}

%%%%%%%%%%%%%%%%%%%%%%%%%%%%%%%%%%%%%%%%%%%%%

\null\vfill
\footnotesize{
  \begin{picture}(0, 0)(0,0) %\begin{picture}(0, 0)(-159, -5)
    \put(0, -20){\includegraphics[scale=0.05]{../../../Images/linkedin-qrcode.png}}
  \end{picture}  \hfill
  LinkedIn: \href{http://www.linkedin.com/pub/xiangzeng-zhou/92/2a0/543}{http://www.linkedin.com/pub/xiangzeng-zhou/92/2a0/54}
}

\pagestyle{myheadings}
%\markright{Xiang-zeng Zhou LinkedIn: \href{http://www.linkedin.com/pub/xiangzeng-zhou/92/2a0/543}{http://www.linkedin.com/pub/xiangzeng-zhou/92/2a0/543}}

%%\XeTeXpdffile ''resume.pdf'' page 1 scaled 800

\end{document}



